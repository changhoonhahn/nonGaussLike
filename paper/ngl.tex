\documentclass[12pt, letterpaper, preprint]{aastex}
\usepackage{hyperref}
%%% This file is generated by the Makefile.
\newcommand{\giturl}{\url{https://github.com/changhoonhahn/nonGaussLike}}
\newcommand{\githash}{0cff07b}\newcommand{\gitdate}{2018-01-09}\newcommand{\gitauthor}{ChangHoon Hahn}


% typesetting shih
\linespread{1.08} % close to 10/13 spacing
\setlength{\parindent}{1.08\baselineskip} % Bringhurst
\setlength{\parskip}{0ex}
\let\oldbibliography\thebibliography % killin' me.
\renewcommand{\thebibliography}[1]{%
  \oldbibliography{#1}%
  \setlength{\itemsep}{0pt}%
  \setlength{\parsep}{0pt}%
  \setlength{\parskip}{0pt}%
  \setlength{\bibsep}{0ex}
  \raggedright
}
\setlength{\footnotesep}{0ex} % seriously?

% math shih
\newcommand{\setof}[1]{\left\{{#1}\right\}}
\newcommand{\given}{\,|\,}
\newcommand{\pseudo}{{\mathrm{pseudo}}}
\newcommand{\Var}{\mathrm{Var}}

% text shih
\newcommand{\foreign}[1]{\textsl{#1}}
\newcommand{\etal}{\foreign{et~al.}}
\newcommand{\opcit}{\foreign{Op.~cit.}}
\newcommand{\documentname}{\textsl{Article}}
\newcommand{\equationname}{equation}
\newcommand{\bitem}{\begin{itemize}}
\newcommand{\eitem}{\end{itemize}}

\begin{document}\sloppy\sloppypar\frenchspacing 

\title{How I Learned to Stop Worrying and Love The Central Limit Theorem}
\date{\texttt{DRAFT~---~\githash~---~\gitdate~---~NOT READY FOR DISTRIBUTION}}
\author{ChangHoon~Hahn\refstepcounter{footnote}\refstepcounter{footnote}, Florian~Beutler, Manodeep~Sinha}
\affil{Lawrence Berkeley National Laboratory, 1 Cyclotron Rd, Berkeley CA 94720, USA}
\email{changhoon.hahn@lbl.gov}

\begin{abstract}
    abstract here 
\end{abstract}

\keywords{
methods: statistical
---
galaxies: statistics
---
methods: data analysis
---
cosmological parameters
---
cosmology: observations
---
large-scale structure of universe
}

\section{Introduction}
\begin{itemize}
    \item Talk about the use of Bayesian parameter inference and getting the posterior in LSS cosmology 
    \item Explain the two major assumptions that go into evaluating the likelihood
    \item Emphasize that we are not talking about non-Gaussian contributions to the likelihood
    \item Emphasize the scope of this paper is to address whether one of the assumptions matters for 
        galaxy clustering analyses. 
\end{itemize}

\section{Gaussian Likelihood Assumption}
\begin{itemize}
    \item Depending on Hogg's paper maybe a simple illustration of how the likelihood asumption 
\end{itemize}

\section{Likelihood non-Gaussianity}
\subsection{Mock Catalogs}
\bitem
    \item brief descriptin of the mock catalogs 
    \item \cite{kitaura2016, sinha2017}
\eitem
\subsection{$k$-NN Nonparametric Divergence Estimation}
\bitem
    \item \cite{poczos2012, krishnamurthy2014}
\eitem

\section{Estimating the Non-Gaussian Likelihood}
\subsection{Nonparametric Likelihood Estimation}
\bitem
    \item Kernel Density Estimation 
    \item Gaussian Mixture Model  
    \item Figure illustrating both methods on highest N GMF bin 
\eitem

\subsection{Independent Component Analysis} 
Curse of dimensionality! 2000 mocks in Beutler not enough to directly estimate 
the 37-dimensional space, so we use independent component analysis 
\cite{hartlap2009}
\bitem
    \item Similar figure to \cite{hartlap2009} that tests the independence? 
\eitem

\section{Impact on Parameter Inference}
\bitem
    \item equations explaining importance sampling framework
    \item details of each of the MCMC runs
\eitem

\section{Discussion}
\bitem
    \item Will it matter for future surveys? 
    \item Likelihood free inference (cite justin's paper) 
\eitem

\section{Summary}

%%%%%%%%%%%%%%%%%%%%%%%%%%%%%%%%%%%%%%%%%%%%%%%%%%%%%%%%%%%%%%%
% Figures 
%%%%%%%%%%%%%%%%%%%%%%%%%%%%%%%%%%%%%%%%%%%%%%%%%%%%%%%%%%%%%%%
\begin{figure}
\begin{center}
\includegraphics[width=0.9\textwidth]{figs/kNN_divergence_gmf.pdf}
\caption{}
\label{fig:gmf_div}
\end{center}
\end{figure}

%%%%%%%%%%%%%%%%%%%%%%%%%%%%%%%%%%%%%%%%%%%%%%%%%%%%%%%%%%%%%%%
% Acknowledgements
%%%%%%%%%%%%%%%%%%%%%%%%%%%%%%%%%%%%%%%%%%%%%%%%%%%%%%%%%%%%%%%
\section*{Acknowledgements}
It's a pleasure to thank 
    Simone~Ferraro,
    David~W.~Hogg,
    Emmaneul~Schaan, 
    Roman~Scoccimarro
    Zachary~Slepian

\bibliographystyle{yahapj}
\bibliography{nongausslike}
\end{document}

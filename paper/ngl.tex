\documentclass[12pt, letterpaper, preprint]{aastex}
\usepackage{hyperref}
%%% This file is generated by the Makefile.
\newcommand{\giturl}{\url{https://github.com/changhoonhahn/nonGaussLike}}
\newcommand{\githash}{891df41}\newcommand{\gitdate}{2017-12-26}\newcommand{\gitauthor}{ChangHoon Hahn}


% typesetting shih
\linespread{1.08} % close to 10/13 spacing
\setlength{\parindent}{1.08\baselineskip} % Bringhurst
\setlength{\parskip}{0ex}
\let\oldbibliography\thebibliography % killin' me.
\renewcommand{\thebibliography}[1]{%
  \oldbibliography{#1}%
  \setlength{\itemsep}{0pt}%
  \setlength{\parsep}{0pt}%
  \setlength{\parskip}{0pt}%
  \setlength{\bibsep}{0ex}
  \raggedright
}
\setlength{\footnotesep}{0ex} % seriously?

% math shih
\newcommand{\setof}[1]{\left\{{#1}\right\}}
\newcommand{\given}{\,|\,}
\newcommand{\pseudo}{{\mathrm{pseudo}}}
\newcommand{\Var}{\mathrm{Var}}

% text shih
\newcommand{\foreign}[1]{\textsl{#1}}
\newcommand{\etal}{\foreign{et~al.}}
\newcommand{\opcit}{\foreign{Op.~cit.}}
\newcommand{\documentname}{\textsl{Article}}
\newcommand{\equationname}{equation}

\begin{document}\sloppy\sloppypar\frenchspacing % oh yeah

\title{How I Learned to Stop Worrying and Love The Central Limit Theorem}
\date{\texttt{DRAFT~---~\githash~---~\gitdate~---~NOT READY FOR DISTRIBUTION}}
\author{ChangHoon~Hahn\refstepcounter{footnote}\refstepcounter{footnote}, Florian~Beutler, Manodeep~Sinha}
\affil{Lawrence Berkeley National Laboratory, 1 Cyclotron Rd, Berkeley CA 94720, USA}
\email{changhoon.hahn@lbl.gov}

\begin{abstract}
    abstract here 
\end{abstract}

\keywords{
methods: data analysis
---
methods: statistical
---
galaxies: statistics
---
cosmological parameters
---
cosmology: observations
---
large-scale structure of universe
}

\section{Introduction}
\begin{itemize}
    \item Talk about the use of Bayesian parameter inference and getting the posterior in LSS cosmology 
    \item Explain the two major assumptions that go into evaluating the likelihood
    \item Emphasize that we are not talking about non-Gaussian contributions to the likelihood
    \item Emphasize the scope of this paper is to address whether one of the assumptions matters for 
        galaxy clustering analyses. 
\end{itemize}

\section{Gaussian Likelihood Assumption}
\begin{itemize}
    \item Depending on Hogg's paper maybe a simple illustration of how the likelihood asumption 
\end{itemize}

\section{Beyond Central Limit Theorem}
\subsection{Quantifying non-Gaussianity}
\begin{itemize}
\item non parameteric $k$-NN estimates of the divergence between the pdf from mock and Gaussian described by the covariance matrix
\end{itemize} 

\subsection{Estimating the Non-Gaussian Likelihood}

\section{Impact on Parameter Inference}
\begin{itemize}
    \item 
\end{itemize}

\acknowledgements

%\bibliography{}

\end{document}
